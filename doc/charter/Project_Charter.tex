%        File: Project_Charter.tex
%     Created: Tue Oct 21 07:00 PM 2014 P
% Last Change: Tue Oct 21 07:00 PM 2014 P
%
\documentclass[letterpaper]{article}
\usepackage[top=0.4in, bottom=1in, left=1in, right=1in]{geometry}
\usepackage{multicol}
%\usepackage{palatino}

\title{Wii Draw: Project Charter}
\author{Team: Hala Diab, Sam Friedman, Joe Wright}
\date{EE C149 Fall 2014}

\begin{document}
\maketitle

\subsection*{Project Goal}
A user will be able to use a WiiMote to draw on a large screen of LEDs,
changing colors depending on drawing mode and sensor input. Drawing modes
could include monotone, color based on how fast the WiiMote is moving, color
based on how the WiiMote is rotated (facing up for blue, down for red, for
example), or user-selectable colors.
\subsection*{Project Approach}
The project will determine the location on the screen at which a user is pointing
a WiiMote by using its accelerometer and infrared camera. Based on other
sensor input, the microcontroller will determine which pixels of the screen to
light up, and what color to make them. Color can be based on user choice,
WiiMote rotation, drawing speed, or other environmental input. Key performance
metrics include pointing accuracy and how easy it feels to use. An initial
focus on a small screen will allow us to first ensure a quality user
experience before scaling up the size of the LED array.
\subsection*{Resources}
The project will use an mbed FRMD KL25Z microcontroller from Freescale for
computation and LED control, a bluetooth board, a
Nintendo WiiMote as the sensor platform (including a 3-axis accelerometer,
infrared camera, and several buttons), NeoPixel (or equivalent)
individually-addressable RGB LEDs in a very large array as the screen, 5V
power supplies sufficient to power the LEDs, and
four clusters of infrared LEDs placed on the edges of the screen for
determining WiiMote orientation. The four LEDs on the WiiMote can be used to
indicate which ``drawing mode'' is active.
\subsection*{Schedule and Project Milestones}
\begin{multicols}{2}
    \noindent Oct 21: Project Charter\\
    Oct 27: Finalize component selection, screen size\\
    Oct 28: Inventory and order components\\
    Nov 4: Complete model extended FSM\\
    Nov 11: WiiMote communication and control\\
    Nov 17: Correct WiiMote pointing\\
    Nov 17: Complete screen construction\\
    Nov 25: Microcontroller and screen communication\\
    Dec 2: Monochrome drawing\\
    Dec 9: Color drawing, drawing modes\\
    Dec 16: Final systems testing and verification\\
    Dec 17: Presentation\\
    Dec 19: Report and Video completed


\end{multicols}

\subsection*{Risk and Feasibility}
Acquiring the components in a timely manner might be tricky, especially if
parts are ordered from overseas. The large number of LEDs will mean that a
significant amount of power is required (each LED can consume up to 0.25W, and
a display may have on the order of thousands of LEDS) so large power supplies
will likely be required (we must also make sure that the screen will not
overload a typical room's power circuit and cause a breaker to flip). Getting
accurate enough WiiMote positioning may be very difficult---even while
using the Wii itself control feels clunky and inaccurate. The mbed hardware
might not have enough memory (16K) to hold the display buffer, but the Teensy 3.1
by PJRC (\texttt{http://www.pjrc.com/teensy/teensy31.html})
is an alternative option with 64K of memory, based on the 72 MHz ARM
Cortex-M4, and is reasonably priced (\$19.80).

\end{document}


